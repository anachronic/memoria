\begin{intro}
	El Departamento de Ciencias de la Computación de la Universidad de Chile
	ofrece distintos programas de magíster. Uno de ellos es el Magíster en
	Tecnologías de la Información (en adelante MTI). Uno de los objetivos de
	este programa es formar especialistas con conocimientos de aspectos
	aplicados respecto al uso, gestión y adopción de Tecnologías de la
	Información y Comunicaciones.

	Teniendo en cuenta el perfil de los profesionales que busca formar este
	programa, y la naturaleza misma del magíster, resulta razonable pensar que
	el proceso de postulación al mismo, así como el procesamiento de dichas
	postulaciones, se haga usando herramientas computacionales.
	
	Hoy en día, la postulación a este programa se hace a través de la plataforma
	UCampus\footnote{UCampus es un Centro Tecnológico que desarrolla plataformas
	de apoyo a la gestión, automatizando los procesos de Instituciones de
	Educación Superior, en particular el de la Universidad de Chile.
	https://ucampus.cl.}. En esa plataforma, el postulante debe ingresar sus
	datos personales, y adjuntar un conjunto de documentos relevantes para
	respaldar su  postulación. Entre esos documentos están las cartas de
	recomendación, el currículum del postulante, su certificado de título y su
	carta motivacional.
	
	Una vez enviada la postulación a través de UCampus, los documentos quedan
	disponibles para que cada programa (en este caso, el MTI) haga con ellos lo
	que estime conveniente. UCampus no ofrece la funcionalidad para apoyar el
	procesamiento de las postulaciones, por lo que cada programa debe hacerlo
	por fuera de esta plataforma. 

	\section{Problema Abordado}

	El proceso de evaluación de postulaciones al MTI se realiza en forma manual.
	Para ello, un funcionario del Programa de Educación Continua (PEC) del DCC
	inicia el proceso descargando manualmente cada uno de los archivos subidos a
	UCampus por el postulante. Esa información luego es enviada a los miembros
	del comité académico del programa a través de un correo electrónico. Los
	miembros de ese comité evalúan las postulaciones. Estas personas pueden
	solicitar información adicional al postulante en caso que se requiera; y en
	esos casos, el funcionario del PEC actúa como intermediario entre el
	postulante y el comité académico.

	Cuando la información de una postulación está completa, los evaluadores emiten
	un juicio individual y justificado respecto a la admisión o no de cada
	postulante al programa. Esta información es enviada vía correo electrónico al
	funcionario del PEC, quien reúne todas las opiniones de los evaluadores acerca
	de una postulación, y luego se las envía al coordinador del programa para que
	decida la admisión o rechazo del postulante. En esa instancia, el coordinador
	revisa manualmente todas las evaluaciones y emite un juicio formal, el cual debe
	ser registrado en la plataforma UCampus, e informado al postulante.

	Este flujo de trabajo (workflow) es completamente informal, por lo que los
	tiempos de procesamiento de una postulación varían mucho, dependiendo de la
	carga de trabajo que tengan los funcionarios del PEC y el resto de los
	participantes. Dado que hoy en día las evaluaciones de las postulaciones se
	realizan de forma manual, el proceso se vuelve lento, costoso y propenso a
	errores. Además, el flujo de trabajo asociado a este proceso es muy difícil de
	monitorear, medir y mejorar. Otro aspecto importante es que este proceso manual
	no escala bien, por lo que se requieren sistemas de apoyo al procesamiento de
	estas postulaciones.

	Tomando en cuenta lo anterior, en esta memoria se diseñó e implementó un sistema Web que:

	\begin{enumerate}
		\item Agiliza el proceso de evaluación de postulaciones al MTI.
		\item Facilita la recopilación y almacenamiento de datos relevantes,
		tanto del postulante como de la evaluación de su postulación.
	\end{enumerate}

	El sistema no busca extender la funcionalidad de UCampus, ni la validez de
	la postulación, por lo tanto, en la memoria se asume que los datos que
	provienen de dicha plataforma son correctos. 

	\section{Objetivos de la Memoria}

	El objetivo general de esta memoria fue desarrollar un sistema Web que
	automatiza parte del flujo de trabajo asociado al procesamiento de
	postulaciones al Magíster en Tecnologías de la Información; programa que es
	impartido por la Escuela de Postgrado de la FCFM\footnote{Facultad de
	Ciencias Físicas y Matemáticas de la Universidad de Chile} a través del
	DCC\footnote{Departamento de Ciencias de la Computación}. Los objetivos
	específicos definidos para alcanzar el objetivo general fueron los
	siguientes:

	\begin{enumerate}
		\item Desarrollar un servicio de software que extraiga automáticamente
		la información de las postulaciones al programa, desde la plataforma
		UCampus.
		\item Desarrollar un sistema que automatice el flujo de trabajo
		(workflow) de evaluación de las postulaciones.
		\item Desarrollar un servicio que facilite la toma de decisiones del
		Coordinador del Programa, respecto a la aceptación o rechazo de los
		postulantes, y a la comunicación de éste con la Escuela de Postgrado.
	\end{enumerate}


	\section{Estructura del Documento}
	**Falta añadir aquí la estructura del documento**.
\end{intro}