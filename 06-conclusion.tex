\begin{conclusion}
	Este trabajo se enmarca en el contexto de la evaluación de postulaciones de
	candidatos a ingresar al programa de Magíster de Tecnologías de la
	Información, que imparte la Escuela de Postgrado a través del DCC. Este
	proceso de evaluación, hasta antes de este trabajo, se realizaba de forma no
	automatizada; particularmente, era un proceso manual apoyado por correo
	electrónico.

	Entendiendo el contexto general en el que se enmarca este trabajo, el
	objetivo general de esta memoria fue desarrollar un sistema Web, que
	automatice parte del flujo de trabajo asociado al procesamiento de las
	postulaciones al programa de Magíster en TI. Para alcanzar ese objetivo, se
	definieron tres objetivos específicos:

	\begin{itemize}
		\item Desarrollar un servicio que extraiga datos de postulaciones desde
		UCampus.
		\item Desarrollar un sistema que asista a los participantes del proceso
		de evaluación en realizar su labor y automatizar el flujo de evaluación
		de cada postulación.
		\item Desarrollar un servicio que facilite la toma de decisiones del
		Coordinador del programa, respecto a la aceptación o rechazo de las
		postulaciones.
	\end{itemize}

	La solución desarrollada para lograr dichos objetivos fue una aplicación web
	que consta de dos componentes: front-end y back-end. El front-end es la
	interfaz con la que interactúan los participantes del proceso, cumpliendo el
	objetivo de asistir a éstos con las tareas que les competen. Por su parte,
	el back-end es el servicio que (1) extrae datos de postulaciones desde
	UCampus, (2) almacena datos tanto de cada postulación como de su historial
	de evaluaciones y su resolución y (3) disponibiliza los datos de una forma
	segura para el consumo de éstos por parte del front-end.

	La solución fue evaluada por los actuales coordinadores del Programa,
	quienes han participado en la evaluación de postulaciones durante al menos
	los últimos 10 años, asumiendo distintos roles en dicho proceso. Durante la
	evaluación estas personas asumieron temporalmente distintos roles (perfiles
	de usuario) para poder realizar el flujo de trabajo completo.

	Como resultado de la evaluación se pudo ver que el software obtenido es
	estable y permite realizar las tareas asignadas a cada rol. También se pudo
	ver que el software también permite la interacción con la plataforma
	UCampus, con las limitaciones mencionadas en el capítulo 2. Estas
	limitaciones se deben a restricciones impuestas por UCampus para cualquier
	software que quiera interactuar con dicha plataforma. A pesar de ello, la
	funcionalidad implementada para la interacción con UCampus reduce
	considerablemente la carga de trabajo y las chances de cometer errores por
	parte de los usuarios asistentes y del coordinador.

	En resumen, se obtuvo un software funcional que puede ser puesto en
	producción (en marcha blanca), para procesar las postulaciones del MTI. Con
	algunos ajustes menores es posible que este software pueda ser utilizado
	para apoyar la evaluación de las postulaciones a otros programas del DCC,
	como por ejemplo, al Magíster en Ciencias y al Doctorado en Computación.

	Como parte del trabajo a futuro, hay dos líneas de trabajo. La primera (y
	más de corto plazo) es abordar los comentarios hechos por los evaluadores
	durante la evaluación del sistema. La segunda línea (de mediano plazo) es
	revisar este software con los coordinadores de otros programas del DCC, a
	fin de identificar funcionalidad extra que estos pudieran requerir. Luego,
	en base a ello, generar un único sistema que (aunque tenga variantes) sirva
	para apoyar a todos los programas.

	%faltan lecciones aprendidas

	
\end{conclusion}
