\chapter{Concepción de la Solución}

En este capítulo se presenta la descripción del sistema desarrollado para dar
solución al problema planteado.

\section{Descripción del Proceso Automatizado}

Habiendo entendido cómo funciona el proceso actual, se puede concebir cómo
automatizar el proceso de postulaciones al MTI. En efecto, el proceso parte
cuando un alumno realiza su postulación en UCampus; ésta se mantiene ahí hasta
que es resuelta en la misma plataforma. Sin embargo, el proceso interno de
resolución del programa MTI no está actualmente tomado en cuenta por UCampus.

Para automatizarlo, se debe partir por extraer los datos desde UCampus. Este
objetivo se logra haciendo un scraper que recorra el sitio, extraiga los datos
de postulaciones y los guarde en una base de datos. Teniendo estos datos, se
puede construir un sistema que asista en el workflow de postulaciones.

Luego, los procesos de validación, evaluación y resolución de la postulación de
cada alumno se realizan actualmente a través de una seguidilla de correos y
recopilación manual de información. Para automatizar este proceso se desarrolló
un sistema que administra el workflow interno de resolución de una postulación
según los requisitos del MTI.

El alcance del sistema desarrollado termina con la evaluación del coordinador,
quien decide si se acepta o rechaza la postulación. Para terminar el workflow
requerido por la Escuela de Postgrado, es necesario que dicho proceso se haga a
través de la plataforma UCampus. Por esa razón, el sistema desarrollado redirige
al coordinador a la postulación en UCampus, para que sea resuelta según el
proceso definido por la Escuela.

\section{Principales Requisitos de la Solución}

A continuación se presentan los principales requisitos de la solución, los
cuales sirvieron de guía para el desarrollo de la misma.

\begin{itemize}
\item Crear un scraper que extraiga datos de postulaciones, tanto nuevas como ya
existentes, desde la plataforma UCampus.
\item Mantener al scraper corriendo periódicamente, de modo que agregue las
postulaciones nuevas y actualice las postulaciones que tienen datos nuevos.
\item Crear una interfaz de usuario que permita validar las postulaciones e
iniciar el proceso de evaluación de éstas.
\item Crear una interfaz de usuario que permita evaluar las postulaciones por
parte de los miembros del comité del programa, y de su coordinador.
\item Crear una interfaz de usuario que permita marcar la resolución de cada
postulación. Se debe redirigir a UCampus para que la postulación siga su proceso
normal en dicha plataforma.
\item Desarrollar un módulo de alertas por correo, que notifique a los
participantes respecto a etapas del workflow que tienen tareas que resolver por
parte de ellos.
\end{itemize}

\section{Perfiles de Usuarios Soportados}

% Necesita refs
En base a los actores del proceso presentados en el capítulo 2 (sección 2.1), se
describe a continuación en qué parte del proceso se involucra cada rol.

\subsection{Postulante}

El postulante no participa activamente del sistema desarrollado. La razón de
esto es porque el objetivo del sistema es automatizar el proceso de evaluación
de las postulaciones, aunque la postulación misma se realiza en UCampus. Este
proceso de evaluación es un workflow externo al se lleva a cabo a través de
UCampus, por lo tanto, el postulante sólo recibe el resultado de su postulación
por parte de la Escuela.

\subsection{Miembro del Comité Académico del Programa}

Los miembros del comité son los encargados de evaluar las postulaciones. Cuando
una postulación es válida (es decir, es correcta y está completa), los miembros
del comité dan su opinión acerca del potencial del postulante, su pregrado, su
experiencia laboral, la carta motivacional enviada, las cartas de recomendación
recibidas y otras observaciones adicionales. Con eso, cada miembro emite una
recomendación de aceptación o rechazo del postulante en una escala de 4 puntos:
aceptar, posible aceptación, posible rechazo, o rechazo. En el Anexo B se
presenta el formulario (en Word) actualmente usado para evaluar estas
postulaciones.
% ^ necesita ref tb

Típicamente una postulación tendrá 4 evaluaciones, una por cada miembro del
comité, pero no necesariamente en todos los casos. En aquellos casos donde hay 3
miembros de acuerdo respecto a la aceptación o rechazo de un candidato, el
coordinador del programa puede decidir sin esperar a que arribe la última
evaluación. Por otra parte, cuando una vez que todos los miembros del comité han
emitido sus recomendaciones sobre un postulante, la postulación pasa
automáticamente a evaluación por parte del coordinador del programa.

\subsection{Coordinador de Magíster / Presidente del Comité Académico}

Cuando todos los miembros del comité emiten su recomendación, el coordinador
hará lo mismo teniendo en cuenta las recomendaciones hechas anteriormente. Esto
se hace para que quede constancia de la opinión del coordinador.

En este paso, el coordinador decide si se acepta o rechaza la postulación. Esta
resolución debe hacerse en la plataforma UCampus, pues ese es el conducto formal
para la Escuela de Postgrado. Debido a eso, el sistema de evaluación
desarrollado abre una ventana nueva con la resolución de la postulación, para
facilitarle la labor al coordinador.

\subsection{Asistente / Funcionario del PEC}

Las postulaciones en UCampus son incrementales. Esto es, pueden estar
almacenadas en forma incompleta, hasta que eventualmente el postulante rellena
todos los datos necesarios para su evaluación, o bien después de un tiempo sin
completarse, el programa decide darle de baja.

Los asistentes o funcionarios del PEC cumplen dos funciones. La primera, es
validar los datos de cada postulación. Vale decir, a pesar de que una
postulación tenga todos los datos necesarios, el sistema no puede verificar
automáticamente que los documentos contengan la información correcta. Este
trabajo lo hacen los asistentes.

La segunda función que cumplen, es monitorear las evaluaciones. Es decir, si una
evaluación lleva demasiado tiempo sin completar, los asistentes les recuerdan a
los evaluadores que deben completar la postulación.
